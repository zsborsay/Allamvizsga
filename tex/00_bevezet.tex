%!TEX root = minta_dolgozat.tex
%%%%%%%%%%%%%%%%%%%%%%%%%%%%%%%%%%%%%%%%%%%%%%%%%%%%%%%%%%%%%%%%%%%%%%%
\chapter*{\textcolor{red}{Bevezető}}
\markboth{Bevezető}{Bevezető}
\addcontentsline{toc}{chapter}{Bevezető}
%%%%%%%%%%%%%%%%%%%%%%%%%%%%%%%%%%%%%%%%%%%%%%%%%%%%%%%%%%%%%%%%%%%%%%%

Napjainkban azt tapasztaljuk, hogy nagyon sok Székelyföldön született szakember dönt a külföldre való kivándorlás mellett. Tanulmányaikat befejezve a jobb élet reményében, vagy a kíváncsiság által hajtva külföldre mennek, ott vállalnak munkát és telepednek le, rövidebb vagy hosszabb időre. Az E-migrated alkalmazás inspirációjául szolgáló projekt, Csala Dénes SZÉKELYDATA című projektje, ezeket a szakembereket kutatja fel, Facebook adatbányászatot használva, és jeleníti meg őket egy világtérképen (\ref{fig:szekely_diaszpora}. ábra~\footnote{Kép forrása: \url{http://csaladenes.egologo.ro/wp-content/uploads/2015/08/szekely_diaszpora12.jpg}, utolsó megtekintés dátuma: 2018-03-07}). Célja az volt, hogy statisztikákat készítsen arról, hogy hol élnek és mivel foglalkoznak az otthonaikat elhagyó, külföldre vándorolt székelyek. 

Az E-migrated alkalmazás tekinthető a SZÉKELYDATA projekt dinamikus, kibővített változataként. A főoldalon levő világtérképen a csatlakozott szakemberek kluszterezett formában jelennek meg és dinamikusan szűrhetőek foglalkozás szerint. A felhasználók módosíthatják lakhelyüket, ezáltal a hozzájuk rendelt térképjelzők az új lakcím szerinti helyüket foglalják el, így naprakészen tartva a térképen  megjelenített statisztikákat. 

Az alkalmazás a Digitális Székelyföld projekt része, amely  a székelyföldi IT Plus Cluster által indított régiófejlesztési kezdeményezés. Az E-migrated projekt célja egy olyan webes szoftverrendszer kialakítása, amely összekapcsolja a külföldön élő, dolgozó szakembereket és lehetővé teszi számukra a szakmai tapasztalatcserét, tudásmegosztást és az egymásnak való segítségnyújtást, kialakítva egy digitális polgárságot. Az alkalmazás felhasználói, olyan külföldön vagy itthon élő, szakmájukban kiemelkedő teljesítményt nyújtó személyek, akik Székelyföldön születtek és szeretnének hozzájárulni szülőföldük technológiai és gazdasági fejlődéséhez. 

A dolgozat további része bemutatja a projekt működését, a felhasznált technológiákat, eszközöket és munkamódszereket.  \Aref{ch:projektrol}. fejezet részletezi az alkalmazás funkcionalitásait, a különböző komponensek közötti kommunikáció megvalósítását, valamint az alkalmazás adatmodelljét.  \Aref{ch:szerver}. és \aref{ch:kliens}. fejezetekben bemutatásra kerülnek a szerver illetve kliens oldali architektúrális megoldások, az implementáció során használt technológiák és eszközök. \Aref{ch:egyebb_eszkozok}. fejezet leírja a fejlesztés során alkalmazott munkamódszereket,  a folyamatos integráció megvalósításához használt technológiákat, a build és függőségkezelő eszközöket, az implementáció során használt fejlesztői környezeteket, valamint a kód minőségét ellenőrző statikus kódelemzőket és teszteket. \Aref{ch:mukodes}. fejezetben kerül sor az alkalmazás működésének szemléltetésére felhasználói szemszögből, különböző ábrák segítségével. Az utolsó fejezet felvet pár továbbfejlesztési lehetőséget és levonja a következtetéseket. 
\begin{figure}
  \centering
  \pgfimage[width=1\linewidth]{images/szekely_diaszpora}
  \caption{A külföldön élő székelyföldiek feltérképezése Facebook bányászattal történt.}
  \label{fig:szekely_diaszpora}
\end{figure}

A projekt tervezése és megvalósítása a Codespring mentroprogram\footnote{Forrás: \url{https://edu.codespring.ro/}, utolsó megtekintés dátuma: 2018-02-27} nyári gyakorlata során kezdődött. Az alkalmazás fejlesztését Tüzes-Bölöni Kincsővel együtt kezdtük el lefektetve az alkalmazás alapjait. A nyár folyamán fokozatosan bővülő specifikáció alapján, egymás után születtek meg az alkalmazás funkcionalitásai. Az egyetemi félév kezdetével, a Csoportos projekt \footnote{Forrás: \url{https://www.cs.ubbcluj.ro/files/curricula/2016/syllabus/IM_sem5_MLM5012_hu_csatol_2016_1984.pdf}, utolsó megtekintés dátuma: 2018-03-16} nevű tantárgy keretein belül, a csapat további három taggal egészült ki, akik szintén hozzájárultak az alkalmazás bővítéséhez. 

A projekt létrejöttéért és a fejlesztés során nyújtott támogatásért köszönet illeti a Codespringet , a mentoraimat, dr. Simon Károly egyetemi adjunktust, Sulyok Csaba doktoranduszt és Szász István szoftverfejlesztőt. Köszönet Szilágyi Zoltán szofterfejlesztőnek a frontend kódbázis alapos átnézéséért és értékeléséért. Nem utolsó sorban pedig köszönet illeti a sok segítségért, támogatásért és biztatásért Tüzes-Bölöni Kincsőt, aki nélkül a projekt nem jöhetett volna létre. 

Az E-migrated alkalmazás bemutatásra került az V. Székelyföldi IT\&C és Innovációs Konferencián \footnote{Forrás: \url{http://www.itpluscluster.ro/hu/node/399}, utolsó megtekintés dátuma: 2018-02-27} és az Erdélyi Tudományos Diákköri Konferencián\footnote{Forrás: \url{https://www.etdk.kmdsz.ro/2018/index.php}, utolsó megtekintés dátuma: 2018-11-07}, ahol dicséretben részesült. Köszönet illeti a konferenciák szervezőit, valamint az aktív résztvevőket, akiknek megjegyzései, hozzászólásai hozzájárultak az alkalmazás funkcionalitásainak bővítéséhez. 


