\chapter{Következtetések és továbbfejlesztési lehetőségek}
\label{ch:kovetkeztetes}

Az E-migrated alkalmazás célja egy olyan közösség kialakítása, melynek tagjai Székelyföldről elszármazott, vagy ott élő szakemberek, akik támogatják és segítik egymást, megosztják tapasztalataikat, tudásukat másokkal, ezáltal hozzájárulva a régió technológia és gazdasági fejlődéséhez. 

A kezdeti nehézségek ellenére sikerült az alapötletnek megfelelő, működő prototípust létrehozni, mely alapjául szolgálhat egy későbbi, éles projektnek is, hiszen a fejlesztés során mindvégig fontos szempont volt az egyszerű bővíthetőség, újrafelhasználhatóság és tesztelhetőség.  A korán kialakított, szilárd alapokon álló architektúrának és az átgondolt design döntéseknek köszönhetően ezt sikerült is megvalósítani. 

A projekt fejlesztése során a funkcionalitások listája folyamatosan bővült és változott. A Scrum munkamódszer alkalmazásából adódóan nem okozott gondot, hogy a projekt indulásakor nem állt rendelkezésre egy kidolgozott specifikáció, hiszen a fokozatosan érkező követelményeket, változásokat gördülékenyen sikerült kezelni. 

Az alkalmazás többszöri bemutatása során a hallgatóságtól érkező visszajelzések és kérdések alapján nagyon sok továbbfejlesztési lehetőség fogalmazódott meg. Ilyen például a bejegyzések téma szerinti rendezése; különböző oktató anyagok, videók közzétételének megvalósítása; teljes, szöveg alapú keresés profiladatok alapján; más szociális hálók Google+, LinkedIn integrálása és a felhasználók egymás közötti kommunikációjának biztosítása. Továbbá felmerült az ötlete egy \textit{gamification} eszköz bevezetésének, amely ösztönözné a felhasználókat, hogy egymáson segítsenek, hiszen a segítségért KÖSZI pontokat kaphatnának, illetve adhatnának, ezáltal nyilvános elismerésben részesítve a jó szándékú személyt. A megszerzett KÖSZI pontok kiváltságokhoz vezethetnének gazdájuk számára, például további meghívókat kaphatnának. 

Végeredményként megfogalmazható, hogy egy alaposan megtervezett és gondosan formált webes alkalmazást sikerült létrehozni, amely hozzájárulhat egy digitális közösség kialakításához, annak fenntartásához, illetve Székelyföld regionális, technológiai és gazdasági fejlődéséhez. Összekapcsolja az egymástól, otthonuktól elszakadt szakembereket, lehetőséget teremtve számukra a szakmai tapasztalatcserére, segítségkérésre és -nyújtásra, valamint az itthon tevékenykedő szakemberekkel való kapcsolatfelvételre, amely potenciálisan hozzájárulhat az emigrált felhasználók hazaköltözéséhez. 
