\begin{abstract}
Napjainkban azt tapasztaljuk, hogy nagyon sok Székelyföldön született szakember dönt a külföldre való kivándorlás mellett. Tanulmányaikat befejezve a jobb élet reményében, vagy a kíváncsiság által hajtva külföldre mennek, ott vállalnak munkát és telepednek le, rövidebb vagy hosszabb időre.  Az E-migrated projekt célja egy olyan webes szoftver-rendszer kialakítása, amely összekapcsolja a külföldön élő, dolgozó szakembereket és lehetővé teszi számukra a szakmai tapasztalatcserét, tudásmegosztást és az egymásnak való segítségnyújtást, kialakítva egy digitális polgárságot. Az alkalmazás felhasználói, olyan külföldön vagy itthon élő, szakmájukban kiemelkedő teljesítményt nyújtó személyek, akik Székelyföldön születtek és szeretnének hozzájárulni szülőföldük technológiai és gazdasági fejlődéséhez.

A dolgozat ismerteti az alkalmazás fejlesztése során használt technológiákat és eszközöket,  bemutatja az architektúrális megoldásokat és bepillantást enged az alkalmazás működésébe, leírva a felhasználók számára elérhető funkcionalitásokat. 

\end{abstract}